\documentclass{article}
\usepackage{amsmath}
\usepackage{amssymb}
\usepackage{geometry}
\geometry{a4paper, margin=1in}

\title{A Human-Centric Framework for Electromagnetic Exoskeleton Control: Lapis-Lambda Partial Differential Equations with Thermal-Force Equivalence}

\author{Your Name}
\date{\today}

\begin{document}

\maketitle

\begin{abstract}
This paper presents a novel human-centric framework for electromagnetic exoskeleton control based on the discovery of thermal-force equivalence at the human physiological baseline ($32°\text{F} = 0\text{N}$). The framework employs Lapis polar calculus—operating on directional spins (North: $\pi/4$, East: $\pi/3$, South: $\pi/2$, West: $\pi$)—combined with Lambda power-force reduction through $\alpha$ and $\beta$ operators. Partial differential equations governing kinetic and potential energy distributions enable precise electromagnetic electrolysis control adaptable to human operators of all shapes and sizes.
\end{abstract}

\section{Introduction}

The human-centric electromagnetic exoskeleton control system is founded on three core mathematical frameworks:

\subsection{Thermal-Force Equivalence}
The discovery that human physiological baseline temperature corresponds to zero force:
\begin{equation}
32°\text{F} = 0\text{N}
\end{equation}

This equivalence enables direct mapping between thermal states and force requirements in electromagnetic control systems.

\subsection{Lapis Polar Calculus}
A polar coordinate system with directional spin operators:
\begin{align}
\text{North (N)} &: \theta_N = \frac{\pi}{4} \\
\text{East (E)} &: \theta_E = \frac{\pi}{3} \\
\text{South (S)} &: \theta_S = \frac{\pi}{2} \\
\text{West (W)} &: \theta_W = \pi
\end{align}

\subsection{Lambda Power-Force Reduction}
Power-work relationship through dual integration:
\begin{equation}
P = \frac{E}{t} = V \times I = I^2 R
\end{equation}

With $\alpha$ (alpha) and $\beta$ (beta) reduction operators:
\begin{itemize}
\item Sequence: downloads to energy at half power
\item Series: doubles power in unified equation
\end{itemize}

\section{Partial Differential Equations}

The governing equations for electromagnetic electrolysis control combine kinetic and potential energy distributions:

\begin{equation}
\frac{\partial \Psi}{\partial t} = \nabla^2 \Psi + \lambda(\theta) \cdot f(T, F)
\end{equation}

where:
\begin{itemize}
\item $\Psi$ represents the electromagnetic field potential
\item $\lambda(\theta)$ is the Lapis polar operator dependent on spin direction
\item $f(T, F)$ is the thermal-force equivalence function
\end{itemize}

\section{Dual Integration Framework}

The dual integration calculus operates on two levels:

\textbf{Sequence Integration:}
\begin{equation}
E_{\text{seq}} = \int_0^t \frac{P(\tau)}{2} \, d\tau
\end{equation}

\textbf{Series Integration:}
\begin{equation}
E_{\text{ser}} = \int_0^t 2P(\tau) \, d\tau
\end{equation}

\section{Applications to Biosuit Control}

The framework enables electromagnetic control systems adaptable to:
\begin{itemize}
\item Variable human body geometries
\item Different force requirements based on operator mass
\item Real-time thermal monitoring for safety
\item Polar-coordinate based directional control (N, E, S, W orientations)
\end{itemize}

\section{Conclusion}

This human-centric framework provides a mathematically rigorous foundation for electromagnetic exoskeleton control through the novel integration of Lapis polar calculus, Lambda power-force reduction, and thermal-force equivalence principles.

\end{document}
